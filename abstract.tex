\chapter*{Abstract}
\addcontentsline{toc}{chapter}{Abstract}
The noisy intermediate-scale quantum (NISQ) era refers to the current technological epoch permeated with quantum processors that are big enough ($50$-$100$ qubits) to be no longer trivially simulatable with digital computers but not yet capable of full fault-tolerant computation. Such processors provide great testbeds to understand the practical issues and resources needed to realize quantum tasks in these processors, such as quantum algorithms. Many pressing issues arise in this context that are a direct consequence of the limitations of these processors (limited number of qubits, low qubit connectivity, and limited coherence times). Hence, for near-term quantum algorithms, there is an overriding imperative to adopt an approach that takes into account, and attempts to mitigate or circumvent some of these limitations.

\bigskip
\noindent
In this thesis, we examine realizing Grover's quantum search algorithm for four qubits on IBM Q superconducting quantum processors, and potentially scaling up to more qubits. We also investigate non-canonical forms of the quantum search algorithm that trade accuracy for speed in a way that is more suitable for near-term processors. Our contribution to this topic of research is a slight improvement in the accuracy of the solution to a graph problem, solved with a quantum search algorithm implemented on IBM Q quantum processors by Satoh et .al in IEEE Transactions on Quantum Engineering (2020). We also explore the realization of a measurement-based quantum search algorithm for three qubits. Unfortunately, the number of qubits and two-qubit gates required by such an algorithm puts it beyond the reach of current quantum processors.

\bigskip
\noindent
Based on a recently published work with Professor Mark Tame, we also report a proof-of-concept demonstration of a quantum order-finding algorithm for factoring the integer $21$. Our demonstration builds upon a previous demonstration by Mart\'in-L\'{o}pez et al. in Nature Photonics 6, 773 (2012). We go beyond this work by implementing the algorithm on IBM Q quantum processors using a configuration of approximate Toffoli gates with residual phase shifts, which preserves its functional correctness and allows us to achieve a complete factoring of $N=21$ using a quantum circuit with relatively fewer two-qubit gates. 

\bigskip
\noindent
Lastly, we realize a small-scale three-qubit quantum processor based on a spontaneous parametric down-conversion source built to generate a polarization-entangled Bell state. The state is enlarged by using the path degree of freedom of one of the photons to make a $3$-qubit GHZ state. The generated state is versatile enough to carry out quantum correlation measurements such as Bell's inequalities and entanglement witnesses. The entire experimental setup is motorized and made automatic allowing remote control of the measurements of each of the qubits, and we design and build a mobile graphical user interface to an provide intuitive and visual way to interact with the experiment.

\chapter*{Abstrak}
\addcontentsline{toc}{chapter}{Abstrak}
Die ruiesende intermediêre skaal kwantum (NISQ) era verwys na die huidige tegnologiese epog deurdring met kwantumverwerkers wat groot genoeg is (50-100 qubits) om nie meer doeltreffend gesimuleer te kan word op digitale rekenaars nie, maar nog nie in staat is om volle foutverdraagsame berekening uit te voer nie. Sulke verwerkers bied baie goeie toetsplatforms om die probleme en hulpbronne mee te verstaan wat nodig is om kwantumtake soos kwantumalgoritmes in hierdie verwerkers te verwesenlik. Baie dringende kwessies ontstaan in hierdie konteks wat 'n direkte gevolg is van die beperkings van hierdie verwerkeers (beperkte aantal qubits, lae qubit konnektiwiteit en beperkte samehang tye). Daarom is daar vir naby-termyn kwantum algoritmes 'n oorheersende noodsaaklikheid om 'n benadering aan te neem wat hierdie beperkings in ag neem en pogings aanwend om sommige daarvan te versag of te omseil.


\bigskip
\noindent
In hierdie handeling het ons ondersoek ingestel na Grover se kwantumsoekalgoritmes vir vier qubits op IBM Q supergeleier kwantumverwerkers en die moontlike opskaal na 'n groter aantal qubits. Ons ondersoek ook nie-kanonieke vorms van die kwantumsoekalgoritmes wat akkuraatheid vir spoed verhandel op 'n manier wat meer geskik is vir naby-termyn verwerkers. Ons bydra tot hierdie navorsingsonderwerp is 'n effense verbetering aan die akkuraatheid van die oplossing vir 'n grafiekprobleem opgelos met 'n soekalgoritme wat op IBM Q kwantumverwerkers geïmplimenteer is deur Satoh et al. In IEEE Transactions on Quantum Engineering (2020). Ons ondersoek ook die verwesenliking van 'n waarneming-gebaseerde kwantumsoekalgoritme vir drie qubits. Die aantal qubits en twee-qubit logikahekke wat deur so 'n algoritme vereis word plaas dit buite die bereik van huidige kwantumverwerkers.


\bigskip
\noindent
Gebaseer op 'n onlangs-gepubliseerde navorsingsstuk saam met professor Mark Tame rapporteer ons ook 'n bewys-van-konsep demonstrasie van 'n kwantum volgordebepaling algoritme vir die faktorisering van die heelgetal 21. Ons demonstrasie bou voort op 'n vorige demonstrasie deur Martín López et al. In Nature Photonics 6,773 (2012). Ons brei uit op hierdie navorsing deur die die algoritme op IBM Q kwantumverwerkers te implimenteer met gebruik van benaderde Toffoli logikahekke met oorblywende faseverskuiwings – wat sy funksionele integriteit behou en ons instaat stel om 'n volledige faktoriseering van N = 21 te bereik met behulp van 'n kwantumstroombaan met 'n kleiner aantal twee-qubit logikahekke.


\bigskip
\noindent
Laastens bewerkstellig ons 'n kleinskaalse drie-qubit kwantumverwerker gebaseer op 'n spontane parametriese fluoressensie (“spontaneous parametric down-conversion”) bron wat gebou is om 'n polarisasie-verstrengelde Bell staat te genereer. Hierdie staat word vergroot deur die baanvryheidsgraad van een van die fotone te gebruik om kwantumkorrelasie metings soos Bell se ongelykhede en verstrengelingsgetuies uit te voer. Die hele eksperimentele opstelling word gemotoriseer en geautomatiseer sodat waarnemings van elk van die qubits deur middel van afstandbeheer gemaak kan word, en ons ontwerp en ontwikkel 'n mobile grafiese gebruikerskoppelvlak om 'n intuïtiewe en visuele manier te bied om met die eksperiment te kommunikeer.

\clearpage
